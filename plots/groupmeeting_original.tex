\documentclass{beamer}

% This file is a solution template for:

% - Talk at a conference/colloquium.
% - Talk length is about 20min.
% - Style is ornate.



% Copyright 2004 by Till Tantau <tantau@users.sourceforge.net>.
%
% In principle, this file can be redistributed and/or modified under
% the terms of the GNU Public License, version 2.
%
% However, this file is supposed to be a template to be modified
% for your own needs. For this reason, if you use this file as a
% template and not specifically distribute it as part of a another
% package/program, I grant the extra permission to freely copy and
% modify this file as you see fit and even to delete this copyright
% notice. 


\mode<presentation> %
{
  %\usetheme{Warsaw}
  % or ...
%\usetheme{Madrid}%https://www.overleaf.com/learn/latex/Beamer#Reference_guide
  \usetheme{AnnArbor}

  \setbeamercovered{transparent}
  % or whatever (possibly just delete it)
}

\usepackage{array}
\usepackage{tabularx} 
\usepackage{underscore}


\usepackage[english]{babel}
% or whatever

\usepackage[latin1]{inputenc}
% or whatever

\usepackage{times}
\usepackage[T1]{fontenc}
% Or whatever. Note that the encoding and the font should match. If T1
% does not look nice, try deleting the line with the fontenc.


\title[IHEP Group Meeting] % (optional, use only with long paper titles)
{Progress Report on Tau Final States of TTTT}
%\subtitle
%{Include Only If Paper Has a Subtitle}
\author[Huiling Hua] % (optional, for multiple authors)
%{A.~B.~Arthur\inst{1} \and J.~Doe\inst{2}}
{Huiling Hua\inst{1} \and Hongbo Liao\inst{1} \and Hideki Okawa\inst{2} \and Yu    Zhang\inst{2}}
%\author{Huiling Hua}
%\institute{IHEP}
\institute[IHEP] % (optional)
{
  \inst{1}%
 % Faculty of Physics\\
 % Very Famous University
    IHEP
  \and
  \inst{2}%
    Fudan University
}
% - Give the names in the same order as the appear in the paper.
% - Use the \inst{?} command only if the authors have different
% - Use the \inst command only if there are several affiliations.
% - Keep it simple, no one is interested in your street address.
\date[IHEP 2020] % (optional, should be abbreviation of conference name)
{IHEP Group Meeting, 2020}
% - Either use conference name or its abbreviation.
% - Not really informative to the audience, more for people (including
%   yourself) who are reading the slides online
\subject{Physics Analysis}
% This is only inserted into the PDF information catalog. Can be left
% out. 

% If you have a file called "university-logo-filename.xxx", where xxx
% is a graphic format that can be processed by latex or pdflatex,
% resp., then you can add a logo as follows:
% \pgfdeclareimage[height=0.5cm]{university-logo}{university-logo-filename}
% \logo{\pgfuseimage{university-logo}}

% Delete this, if you do not want the table of contents to pop up at
% the beginning of each subsection:
\AtBeginSubsection[]
{
  \begin{frame}<beamer>{Outline}
    \tableofcontents[currentsection,currentsubsection]
  \end{frame}
}

% If you wish to uncover everything in a step-wise fashion, uncomment
% the following command: 
%\beamerdefaultoverlayspecification{<+->}


\begin{document}

\begin{frame}
  \titlepage
\end{frame}

\begin{frame}{Outline}
  \tableofcontents
  % You might wish to add the option [pausesections]
\end{frame}


% Structuring a talk is a difficult task and the following structure
% may not be suitable. Here are some rules that apply for this
% solution: 

% - Exactly two or three sections (other than the summary).
% - At *most* three subsections per section.
% - Talk about 30s to 2min per frame. So there should be between about
%   15 and 30 frames, all told.

% - A conference audience is likely to know very little of what you
%   are going to talk about. So *simplify*!
% - In a 20min talk, getting the main ideas across is hard
%   enough. Leave out details, even if it means being less precise than
%   you think necessary.
% - If you omit details that are vital to the proof/implementation,
%   just say so once. Everybody will be happy with that.

\section{Motivation}
\subsection{Introduction to 4 Tops Process}

%\begin{frame}{Make Titles Informative. Use Uppercase Letters.}{Subtitles are optional.}
\begin{frame}{4 Tops Process}%{Sunbtitles are optional.}
  % - A title should summarize the slide in an understandable fashion
  %   for anyone how does not follow everything on the slide itself.
  \begin{itemize}
  \item
    1% Use \texttt{itemize} a lot.
  \item
    2%Use very short sentences or short phrases.
  \end{itemize}
\end{frame}


\subsection{Previous Work}
\begin{frame}{Make Titles Informative.}
\end{frame}


\section{CutBased Selection}
\begin{frame}{Data and MC samples}
    \begin{itemize}
    \item
    1
    \end{itemize}
    \begin{table}[htbp] %[h]
    \centering
    \small
    \setlength\tabcolsep{2pt}
    \begin{tabular}{|l | l | c|} 
     \hline
     Process & Sample Name & Cross Section  \\% [0.5ex] %\[1ex]This adds extra space to the cell
     \hline\hline
    TTTT & TTTT_TuneCUETP8M2T4_13TeV-amcatnlo-pythia8 & 00000 \\

TTJets_TuneCUETP8M2T4_13TeV-amcatnloFXFX-pythia8.root    %7.467e+02 +- 2.820e+00 pb
TTGJets_TuneCUETP8M1_13TeV-amcatnloFXFX-madspin-pythia8       %3.773e+00 +- 1.178e-02 pb
ttZJets_13TeV_madgraphMLM-pythia8.root       %6.559e-01 +- 5.438e-04 p
ttWJets_13TeV_madgraphMLM.root       %
ttH_4f_ctcvcp_TuneCP5_13TeV_madgraph_pythia8   %
ttbb_4FS_ckm_amcatnlo_madspin_pythia8.root       %1.393e+01 +- 3.629e-02 pb
WZ_TuneCUETP8M1_13TeV-pythia8.root      %2.343e+01 +- 1.049e-02 pb
WW_TuneCUETP8M1_13TeV-pythia8.root       %6.430e+01 +- 2.817e-02 pb
WpWpJJ_EWK-QCD_TuneCUETP8M1_13TeV-madgraph-pythia8.root       %5.390e-02 +- 2.905e-05 pb
ZZ_TuneCUETP8M1_13TeV-pythia8       %1.016e+01 +- 5.141e-03 pb
WGJets_MonoPhoton_PtG-40to130_TuneCUETP8M1_13TeV-madgraph.root       %
ZGJetsToLLG_EW_LO_13TeV-sherpa.root       %
WWW_4F_TuneCUETP8M1_13TeV-amcatnlo-pythia8.root        %
WWZ_TuneCUETP8M1_13TeV-amcatnlo-pythia8       %
WWG_TuneCUETP8M1_13TeV-amcatnlo-pythia8.root       %
ZZZ_TuneCUETP8M1_13TeV-amcatnlo-pythia8.root       %
WZZ_TuneCUETP8M1_13TeV-amcatnlo-pythia8.root       %
WZG_TuneCUETP8M1_13TeV-amcatnlo-pythia8.root       %
WGG_5f_TuneCUETP8M1_13TeV-amcatnlo-pythia8.root       %
WGGJets_TuneCUETP8M1_13TeV_madgraphMLM_pythia8       %
ZGGJets_ZToHadOrNu_5f_LO_madgraph_pythia8.root       %
WJetsToLNu_TuneCUETP8M1_13TeV-madgraphMLM-pythia8       %
DYJetsToTauTau_ForcedMuEleDecay_M-50_TuneCUETP8M1_13TeV-amcatnloFXFX-pythia8_ext1      %
tZq_ll_4f_ckm_NLO_TuneCP5_PSweights_13TeV-amcatnlo-pythia8.root       %
tZq_nunu_4f_13TeV-amcatnlo-pythia8_TuneCUETP8M1.root       %
ST_tW_antitop_5f_inclusiveDecays_13TeV-powheg-pythia8_TuneCUETP8M2T4       %
ST_tW_top_5f_inclusiveDecays_13TeV-powheg-pythia8_TuneCUETP8M2T4       %
TGJets_TuneCUETP8M1_13TeV_amcatnlo_madspin_pythia8.root       %
THW_ctcvcp_HIncl_M125_TuneCP5_13TeV-madgraph-pythia8   %
THQ_ctcvcp_Hincl_13TeV-madgraph-pythia8_TuneCUETP8M1   %
VHToNonbb_M125_13TeV_amcatnloFXFX_madspin_pythia8   %
ZHToTauTau_M125_13TeV_powheg_pythia8   %
ZH_HToBB_ZToLL_M125_13TeV_powheg_pythia8   %
GluGluHToZZTo4L_M125_13TeV_powheg2_JHUgenV6_pythia8   %
GluGluHToBB_M125_13TeV_amcatnloFXFX_pythia8   %
GluGluHToGG_M125_13TeV_amcatnloFXFX_pythia8   %
GluGluHToMuMu_M-125_TuneCP5_PSweights_13TeV_powheg_pythia8   %
GluGluHToTauTau_M125_13TeV_powheg_pythia8   %
GluGluHToWWTo2L2Nu_M125_13TeV_powheg_JHUgen_pythia8   %
GluGluHToWWToLNuQQ_M125_13TeV_powheg_JHUGenV628_pythia8   %
VBFHToWWToLNuQQ_M125_13TeV_powheg_JHUGenV628_pythia8   %
VBFHToWWTo2L2Nu_M125_13TeV_powheg_JHUgenv628_pythia8   %
VBFHToTauTau_M125_13TeV_powheg_pythia8   %
VBFHToMuMu_M-125_TuneCP5_PSweights_13TeV_powheg_pythia8   %
VBFHToGG_M125_13TeV_amcatnlo_pythia8_v2   %
VBFHToBB_M-125_13TeV_powheg_pythia8_weightfix   %
VBF_HToZZTo4L_M125_13TeV_powheg2_JHUgenV6_pythia8   %



     \hline
    \end{tabular}
    \caption{Table to test captions and labels}
    \label{table:1}
    \end{table}   
\end{frame}




\subsection{Selection Without Categorization}
\begin{frame}{Object Selection}
  \begin{itemize}
  \item
    Electron 
    \begin{itemize}
    \item
        pt>20, |eta|<2.4
    \item
        electron cut based loose ID(Fall-94X-V2)
    \end{itemize}
  \item
    Muon
    \begin{itemize}
    \item
        pt>20, |eta|<2.4
    \item
        loose ID (cutbased, recommended by muon POG)
    \item
        pass loose isolation(same as SS of TTTT)
    \end{itemize}
  \item
    Tau
    \begin{itemize}
    \item
        pt>20, |eta|<2.3
    \item
        loose tau ID(same as ttH)
    \end{itemize}
  \end{itemize}
\end{frame}


\begin{frame}{Object Selection}
  \begin{itemize}
  \item
    Jet
    \begin{itemize}
    \item
        pt>25
    \item
        loose jet(recommended by JETMET)
    \end{itemize}
  \item
     B Jet
     \begin{itemize}
        \item
            use Deep Flavour B tagging algorithm 
        \item 
            use the recommended working points
     \end{itemize}
  \item
      Top 
    \begin{itemize}
    \item
        use SUSY HOT TopTagger 
    \item 
        resolved
    \end{itemize}

  \end{itemize}
\end{frame}


\begin{frame}{Event Selection-Preselection}
    \begin{itemize}
    \item
        MET filters
    \item
        at least 1 loose tau
    \end{itemize}
\end{frame}


\begin{frame}{Distribution of Signal Vs Background}
    \begin{itemize}
    \item
      1 
    \item
       d 
    \end{itemize}
\end{frame}



\begin{frame}{Significance}
    \begin{itemize}
    \item
       d
    \item
        d
    \end{itemize}
\end{frame}



\begin{frame}{Optimazation}
    \begin{itemize}
    \item
       d
    \item
        d
    \end{itemize}
\end{frame}


\subsection{Selection With Categorization}


\begin{frame}{}
    \begin{itemize}
    \item
       d
    \item
        d
    \end{itemize}
\end{frame}

\begin{frame}{Make Titles Informative.}
    \begin{itemize}
    \item
       d
    \item
        d
    \end{itemize}
\end{frame}

\begin{frame}{Make Titles Informative.}
    \begin{itemize}
    \item
       d
    \item
        d
    \end{itemize}
\end{frame}



\section*{Summary}

\begin{frame}{Summary}
  % Keep the summary *very short*.
  \begin{itemize}
  \item
    The \alert{first main message} of your talk in one or two lines.
  \item
    The \alert{second main message} of your talk in one or two lines.
  \item
    Perhaps a \alert{third message}, but not more than that.
  \end{itemize}
  % The following outlook is optional.
  \vskip0pt plus.5fill
  \begin{itemize}
  \item
    Outlook
    \begin{itemize}
    \item
      Something you haven't solved.
    \item
      Something else you haven't solved.
    \end{itemize}
  \end{itemize}
\end{frame}

\begin{frame}
\frametitle{Sample frame title}

In this slide, some important text will be
\alert{highlighted} because it's important.
Please, don't abuse it.

\begin{block}{Remark}
Sample text
\end{block}

\begin{alertblock}{Important theorem}
Sample text in red box
\end{alertblock}

\begin{examples}
Sample text in green box. The title of the block is ``Examples".
\end{examples}
\end{frame}

\begin{frame}{Make Titles Informative.}
  You can create overlays\dots
  \begin{itemize}
  \item using the \texttt{pause} command:
    \begin{itemize}
    \item
      First item.
      \pause
    \item    
      Second item.
    \end{itemize}
  \item
    using overlay specifications:
    \begin{itemize}
    \item<3->
      First item.
    \item<4->
      Second item.
    \end{itemize}
  \item
    using the general \texttt{uncover} command:
    \begin{itemize}
      \uncover<5->{\item
        First item.}
      \uncover<6->{\item
        Second item.}
    \end{itemize}
  \end{itemize}
\end{frame}


% All of the following is optional and typically not needed. 
\appendix
\section<presentation>*{\appendixname}
\subsection<presentation>*{For Further Reading}

\begin{frame}[allowframebreaks]
  \frametitle<presentation>{For Further Reading}
    
  \begin{thebibliography}{10}
    
  \beamertemplatebookbibitems
  % Start with overview books.

  \bibitem{Author1990}
    A.~Author.
    \newblock {\em Handbook of Everything}.
    \newblock Some Press, 1990.
 
    
  \beamertemplatearticlebibitems
  % Followed by interesting articles. Keep the list short. 

  \bibitem{Someone2000}
    S.~Someone.
    \newblock On this and that.
    \newblock {\em Journal of This and That}, 2(1):50--100,
    2000.
  \end{thebibliography}
\end{frame}

\end{document}


